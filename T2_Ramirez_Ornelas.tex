\documentclass[10pt,a4paper]{article}
\usepackage[T1]{fontenc}
\title{Development and Execution Tools for PWAs}
\author{Ramirez Ornelas Eric Sebastian}
\begin{document}
	\maketitle
	\textbf{Tools}
	\begin{itemize}
		\item Visual Studio Code: A highly configurable and extensible source code editor that is popular among web developers.
		\item Chrome DevTools: Integrated tool in the Google Chrome browser that facilitates development, debugging, and profiling of web applications.
		\item Lighthouse: An open-source automated tool for improving the quality of PWAs. It can perform audits for performance, accessibility, best practices, and SEO.
		\item Webpack: A module bundler that helps structure and optimize application code. It is commonly used in PWA projects for creating efficient bundles.
	\end{itemize}
	
	\textbf{Installation Requirements}
	\begin{itemize}
		\item Security: A secure connection (HTTPS) for your site ensures that all traffic is as secure as a native app. A secure endpoint also allows the service worker to take secure actions on behalf of your application.
		\item Standard Manifest: The site must be controlled by a W3C manifest that defines the experience and behavior of your PWA. This includes everything from images to language to the home page of your web application.
		\item Independence: The progressive web app must have a Network-Independent mechanism (e.g., through a service worker) to help manage traffic when the network is not available or unreliable. The application should be able to function independently of the network.
		\item Responsiveness: The site should be responsive on tablets and mobile devices.
		\item Cross-Browser: The site should work on various browsers (e.g., Chrome, Edge, Firefox, and Safari).
		\item Deep Linking: Each page of the site should have a unique URL (individual pages can be deep-linked via URLs, e.g., for sharing on social media).
	\end{itemize}
\end{document}