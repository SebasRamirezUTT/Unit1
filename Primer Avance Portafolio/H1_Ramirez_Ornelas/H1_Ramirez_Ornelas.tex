\documentclass[10pt,a4paper]{article}
\usepackage[T1]{fontenc}
\title{Concepts and Characteristics of PWAs}
\author{Ramirez Ornelas Eric Sebastian}
\begin{document}
	\maketitle
	\textbf{Concept}\\
	Progressive Web Apps (PWAs) are web applications that progressively adapt to the capabilities of the browser and device. These applications are accessible through a standard web browser, eliminating the need for downloads from an app store.\\
	PWAs offer an experience similar to native apps, being responsive, fast, and capable of accessing device features while requiring less storage and memory resources.\\
	
	\textbf{What is the difference between regular apps, PWAs, and traditional websites?}\\
	The main difference between regular apps, Progressive Web Apps (PWAs), and traditional websites lies in their capabilities.
	
	Compared to traditional websites, regular mobile apps launch quickly and tend to load even with low or no network availability. These apps and PWAs can send notifications and share location, advantages that the web had not offered to users until now. PWAs have the same capabilities as a mobile app and enhance web usage but do not require mobile storage space for app downloads.\\
	
	\textbf{Service and Progressive Applications}\\
	Service-oriented architecture is one of the most popular alternatives to the traditional way of building applications in the form of a monolith.\\
	
	Progressive Web App (PWA) architecture is based on Single Page Application architecture, providing offline capabilities for your web application. Technologies like Capacitor and Ionic are used to create PWAs that can offer users a seamless experience across all platforms.\\
	
	\textbf{Characteristics}
	\begin{itemize}
		\item Designed to adapt to different screen sizes.
		\item Can function even when the user is not connected to the internet.
		\item Automatically update.
		\item To ensure security, PWAs require secure connections using the HTTPS protocol.
	\end{itemize}
	
	\textbf{Advantages}
	\begin{itemize}
		\item Multiplatform, accessible from any browser and operating system.
		\item Adapt their functions to the user's browser.
		\item Allow informing users through push notifications.
		\item Do not need to be downloaded, thus not occupying space on the mobile device.
	\end{itemize}
	
	\textbf{Disadvantages}
	\begin{itemize}
		\item Cannot access all device functionalities, such as contact lists or advanced camera utilities.
		\item Currently, not all browsers are compatible with PWAs, so a portion of users still cannot access them.
		\item PWAs are suitable for developing a mobile-friendly and accessible app, but their technology does not yet allow for creating very complex sites.
	\end{itemize}
	Urrutia, D. (2023, 16 octubre). Qué es aplicación web progresiva (PWA) - definición. Arimetrics. https://www.arimetrics.com/glosario-digital/aplicacion-web-progresiva-pwa\\
	
	Sana Commerce. (2023, 11 mayo). ¿Qué es PWA? | Conceptos de comercio electrónico. https://www.sana-commerce.com/es/conceptos-de-comercio-electronico/que-es-pwa\\
	
	Harsh, K. (2022, 14 octubre). ¿Qué es la arquitectura de las aplicaciones web? Desglosando una aplicación web. Kinsta®. https://kinsta.com/es/blog/arquitectura-aplicaciones-web/\\
	
	Herrera, I. Herrera, I. (2023, 27 julio). Ventajas y desventajas de las PWA. ttandem.com. https://www.ttandem.com/blog/desarrollo-que-son-las-pwa-o-progressive-web-applications/ventajas-y-desventajas-de-las-pwa/
\end{document}
	

\end{document}